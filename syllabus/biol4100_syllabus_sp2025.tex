\documentclass[12pt, notitlepage]{article}   	% use "amsart" instead of "article" for AMSLaTeX format
\usepackage{geometry}                		% See geometry.pdf to learn the layout options. There are lots.
\geometry{a4paper}                   		% ... or a4paper or a5paper or ... 
%\geometry{landscape}                		% Activate for rotated page geometry
\usepackage[parfill]{parskip}    		% Activate to begin paragraphs with an empty line rather than an indent
\usepackage{graphicx}				% Use pdf, png, jpg, or eps§ with pdflatex; use eps in DVI mode
								% TeX will automatically convert eps --> pdf in pdflatex

\usepackage{hyperref}
		
%SetFonts

\usepackage[T1]{fontenc}
\usepackage[utf8]{inputenc}

\usepackage{tgbonum}

%SetFonts

\title{
	\textbf{
		BIOL 4100-029
	} \\
	\large Undergraduate Research \\
	\large Spring 2025
}

\date{\vspace{-5ex}}

\begin{document}

{\fontfamily{phv}\selectfont %select helvetica (code = phv)

\maketitle

\section{Course Description}
This course focuses on the approaches, techniques, and methodologies required to examine plant physiological and terrestrial ecosystem responses and feedbacks to the environment. The specific tools taught will be dependent on student interest, but may include literature review, experimental design, measurements of plant gas exchange and environmental manipulation, regional- and global-scale modeling, as well as large data extraction, analysis, and dissemination. The course is designed to be broad and catered to student interest; however, some readings will tend to focus on an ongoing project in the Smith Lab focused on understanding the effects of plant invasion on native populations.

\section{Expected Learning Outcomes and Objectives}
Upon completion of this class, students are expected to be able to:\par
(1)	Read, review, and discuss past and present scientific literature\par
(2) Design a scientific experiment(s)\par
(3) Disseminate scientific ideas\par

\section{Responsibilities of the Student}
Each student is expected to:\par
(1) Perform ~1-2 hours of lab work per week\par
(2) Attend the 1-hour small group meeting each week (time TBD)\par
(3) Maintain an electronic journal\par
(4) Adhere to the safety instructions at all times\par
(5) Attend the 1-hour lab meeting each week if your course schedule allows (time TBD)\par

\subsection{Class Time and Location}
Class time: Wednesday 3-4 pm
Class location: Experimental Sciences Building II (ESBII) Room 409A


\subsection{Instructor}
Dr. Evan Perkowski \par
ESB II Room 402D \par
717-331-4303 \par
evan.a.perkowski@ttu.edu \par 
\textit{Meetings by appointment}

\subsection{Recommended Texts}
Plant Physiological Ecology (2nd Edition; 2008) by Lambers, Chapin, and Pons \par
The book can be accessed from Springer here: 
\url{https://www.springer.com/us/book/9780387783406}. Click on "Access this title on 
SpringerLink." It can also be accessed through the TTU library. \par
Plant Physiology and Development (6th Edition) by Taiz, Ziegler, Moller, and Murphy

\section{Mode of Instruction}
All instruction will be done face-to-face unless the university directs classes be 
taught online (see next section).

\section{Contingency Statement}
This course is being taught primarily in the face-to-face learning mode. The University will continue to monitor CDC, State, and TTU System guidelines in continuing to manage the campus implications of COVID-19. Any changes affecting class policies or delivery modality will be in accordance with those guidelines and announced as soon as possible. If Texas Tech University campus operations are required to change because of health concerns related to the COVID-19 pandemic, it is possible that this course will move to a fully online delivery format. Should that be necessary, students will need to have access to the Internet, a webcam, and microphone for remote delivery of the class. 

\section{Course Materials}
All course materials, including lecture slides, readings, activities, and code will be posted to a GitHub repository for the course.
The primary repository address is \url{https://github.com/SmithEcophysLab/biol4100_fall2022}.

\section{Attendance Policy}
Attendance is strongly recommended. Course assessments will be done during class (see below).

\section{Course Assessment}
\subsection{\textit{Participation and Engagement}}
Being an active and engaged participant in the class will benefit your understanding of material as well as your peers'. Examples include asking questions, providing feedback, and facilitating discussion. Participation and engagement of each student will be monitored during each class period.

\subsection{\textit{Lab journal}}
Every Friday after 4 pm, your electronic lab journal will be checked to ensure that it is up-to-date. The content of the lab journal will vary by week. Each week, a brief oral report of work done is required. In weeks where data is taken, methods, data, and metadata should be uploaded to the lab journal. Enough information must be written in the laboratory journal to enable independent reproduction and use of the data.

\subsection{\textit{Final report}}
The final report will consist of a project proposal for a project to be carried out in a subsequent semester.

\section{Grading}
Participation and Engagement: 50\% \par
Lab journal: 25\% \par
Final report: 25\% \par

\section{Grading Scale}
A: $\geq$ 90\% \par
B: 80 – 90\% \par
C: 70 – 80\% \par
D: 60 – 70\% \par
F: $\leq$ 59.9\% \par

\section{Missing In-class Activities}
Students will be required to be in class to receive in-class activity points. Please contact Dr. Smith if you plan to miss class for a university function \textit{prior to class}. If class is missed due to an illness, please let Dr. Perkowski know as soon as possible (see COVID illness based absence policy below).

\section{Special Considerations}
\subsection{ADA Statement}
Any student who, because of a disability, may require special arrangements in order to meet the course requirements should contact the instructor as soon as possible to make any necessary arrangements. Students should present appropriate verification from Student Disability Services during the instructor’s office hours. Please note: instructors are not allowed to provide classroom accommodations to a student until appropriate verification from Student Disability Services has been provided. For additional information, please contact Student Disability Services in Weeks Hall or call 806-742-2405.


\section{Academic Integrity Statement}
Academic integrity is taking responsibility for one’s own class and/or course work, being individually accountable, and demonstrating intellectual honesty and ethical behavior. Academic integrity is a personal choice to abide by the standards of intellectual honesty and responsibility. Because education is a shared effort to achieve learning through the exchange of ideas, students, faculty, and staff have the collective responsibility to build mutual trust and respect. Ethical behavior and independent thought are essential for the highest level of academic achievement, which then must be measured. Academic achievement includes scholarship, teaching, and learning, all of which are shared endeavors. Grades are a device used to quantify the successful accumulation of knowledge through learning. Adhering to the standards of academic integrity ensures grades are earned honestly. Academic integrity is the foundation upon which students, faculty, and staff build their educational and professional careers. [Texas Tech University (“University”) Quality Enhancement Plan, Academic Integrity Task Force, 2010].

 \section{Religious Holy Day Statement}
"Religious holy day" means a holy day observed by a religion whose places of worship are exempt from property taxation under Texas Tax Code §11.20. A student who intends to observe a religious holy day should make that intention known in writing to the instructor prior to the absence. A student who is absent from classes for the observance of a religious holy day shall be allowed to take an examination or complete an assignment scheduled for that day within a reasonable time after the absence. A student who is excused under section 2 may not be penalized for the absence; however, the instructor may respond appropriately if the student fails to complete the assignment satisfactorily.


\section{Plagiarism Statement}
Texas Tech University expects students to “understand the principles of academic integrityand abide by them in all class and/or course work at the University” (OP 34.12.5). Plagiarism is a form of academic misconduct that involves (1) the representation of words,  ideas, illustrations, structure, computer code, other expression, or media of another as one's own and/or failing to properly cite direct, paraphrased, or summarized materials; or (2) self-plagiarism, which involves the submission of the same academic work more than once without the prior permission of the instructor and/or failure to correctly cite previous work written by the same student. Please review Section B of the TTU Student Handbook for more information related to other forms of academic misconduct, and contact your instructor if you have questions about plagiarism or other academic concerns in your courses. To learn more about the importance of academic integrity and practical tips for avoiding plagiarism, explore the resources provided by the TTU Library and the School of Law.

\section{Discrimination, Harassment, and Sexual Violence Statement}
Texas Tech University is committed to providing and strengthening an educational, working, and living environment where students, faculty, staff, and visitors are free from gender and/or sex discrimination of any kind. Sexual assault, discrimination, harassment, and other Title IX violations are not tolerated by the University. Report any incidents to the Office for Student Rights & Resolution, (806)-742-SAFE (7233) or file a report online through the Title IX office. Faculty and staff members at TTU are committed to connecting you to resources on campus. Some of these available resources are:
\begin{itemize}
    \item TTU Student Counseling Center, 806- 742-3674: Provides confidential support on campus
    \item TTU 24-hour Crisis Helpline, 806-742-5555: Assists students who are experiencing a mental health or interpersonal violence crisis. If you call the helpline, you will speak with a mental health counselor.
    \item Voice of Hope, 806-763-7273: 24-hour hotline that provides support for survivors of sexual violence.
    \item Risk, Intervention, Safety and Education (RISE) Office, 806-742-2110: Provides a range of resources and support options focused on prevention education and student wellness.
    \item Texas Tech Police Department, 806-742- 3931: To report criminal activity that occurs on or near Texas Tech campus.
\end{itemize}

\section{AI Use}
The use of generative AI tools (such as ChatGPT) is strictly prohibited in this course for any purpose.
Information gathered from AI cannot be used even with appropriate citation. Submission of AI-generated
content (i.e., information, text, or images) as your own work is a violation of academic integrity and may
result in referral to the Office of Student Conduct. Please contact your instructor if you have questions
regarding this course policy.

\section{Statement about Food Insecurity}
Any student who faces challenges securing their food or housing
and believes this may affect their performance in the course is
urged to contact the Dean of Students for support. Furthermore,
please notify the professor if you are comfortable in doing so. The
TTU Food Pantry is in Doak Hall room 117. Please visit the
website for hours of operation at \url{https://www.depts.ttu.edu/dos/foodpantry.php}.


\newpage

\section*{Schedule of Topics by Week}
*subject to change based on student interests* \par
08/28/2023 – Introductions, semester planning, and goals \par
09/04/2023 – Plants and ecosystem services \par
09/11/2023 – Introduction to the Nutrient Network \par
09/18/2023 – Nutrient Network visit and sampling \par
09/25/2023 – Eutrophication and biomass \par
10/02/2023 – Eutrophication and plant diversity \par
10/09/2023 – Eutrophication and leaf traits \par
10/16/2023 – Eutrophication and gas exchange \par
10/23/2023 – Eutrophication and herbivory \par
10/30/2023 – Eutrophication and soil processes \par
11/06/2023 – Proposal writing brainstorm \par
11/20/2023 – NO CLASS \par
11/27/2023 – Proposal presentations \par
12/05/2023 – Proposal presentations \par

} %end font selection

\end{document} 
