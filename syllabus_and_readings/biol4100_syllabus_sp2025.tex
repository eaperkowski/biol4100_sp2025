\documentclass[12pt, notitlepage]{article}   	% use "amsart" instead of "article" for AMSLaTeX format
\usepackage{geometry}                		% See geometry.pdf to learn the layout options. There are lots.
\geometry{a4paper}                   		% ... or a4paper or a5paper or ... 
%\geometry{landscape}                		% Activate for rotated page geometry
\usepackage[parfill]{parskip}    		% Activate to begin paragraphs with an empty line rather than an indent
\usepackage{graphicx}				% Use pdf, png, jpg, or eps§ with pdflatex; use eps in DVI mode
								% TeX will automatically convert eps --> pdf in pdflatex

\usepackage{hyperref}
		
%SetFonts

\usepackage[T1]{fontenc}
\usepackage[utf8]{inputenc}

\usepackage{tgbonum}

%SetFonts

\title{
	\textbf{
		BIOL 4100-029
	} \\
	\large Undergraduate Research \\
	\large Spring 2025
}

\date{\vspace{-5ex}}

\begin{document}

{\fontfamily{phv}\selectfont %select helvetica (code = phv)

\maketitle

\section{Course Description}
This course focuses on understanding the approaches and techniques needed to examine plant physiological responses and feedbacks to the environment. The specific tools taught will be dependent on student interest and decided during the first class meeting period, but may include literature review, experimental design, and measurements of plant gas exchange and environmental manipulation. Students will also learn about best practices for data cleaning and management, analysis, and dissemination. The course is designed to be broad and catered to student interest; however, weekly readings will be related to ongoing projects in the Smith Plant Ecophysiology Lab focused on understanding plant ecophysiological responses to disturbance (e.g., due to climate change or plant invasion). 

\section{Prerequisites}
None.

\section{Expected Learning Outcomes and Objectives}
Upon completion of this class, students are expected to be able to:\par
(1)	Read, review, and discuss past and present scientific literature\par
(2) Design a scientific experiment(s)\par
(3) Disseminate scientific ideas\par

\section{Responsibilities of the Student}
Each student is expected to:\par
(1) Perform at least 1-2 hours of lab work per week\par
(2) Attend the 1-hour small group meeting each week (Wed 3-4pm)\par
(3) Maintain an electronic journal\par
(4) Adhere to the safety instructions at all times\par
(5) Attend the 1-hour lab meeting each week if your course schedule allows (time TBD)\par

\subsection{Class Time and Location}
Class time: Wednesday 3-4 pm\par
Class location: Experimental Sciences Building II (ESB II) Room 409A

\subsection{Instructor}
Dr. Evan Perkowski \par
ESB II Room 402 \par
\href{mailto:evan.a.perkowski@ttu.edu}{evan.a.perkowski@ttu.edu} \par 
\textit{Office hours: Wednesday 2-3pm in ESB II Room 409D. Please e-mail Dr. Perkowski to schedule meetings outside of office hours.}

\subsection{Recommended Texts}
Plant Physiological Ecology (2nd Edition; 2008) by Lambers, Chapin, and Pons \par
The book can be accessed from Springer here: \url{https://www.springer.com/us/book/9780387783406}. Click on "Access this title on SpringerLink." It can also be accessed through the TTU library. \par
Plant Physiology and Development (6th Edition) by Taiz, Ziegler, Moller, and Murphy

\section{Mode of Instruction}
All instruction will be done face-to-face unless otherwise noted.

\section{Course Materials}
All course materials, including readings, activities, and code will be posted to a GitHub repository for the course. The repository can be found here: 
\url{https://github.com/eaperkowski/biol4100_sp2025}

\section{Attendance Policy}
Attendance is strongly recommended. Course assessments will be done during class (see below).

\section{Course Assessment}
\subsection{\textit{Participation and Engagement}}
Being an active and engaged participant in the class will benefit your understanding of material as well as your peers'. Examples include asking questions, providing feedback, and facilitating discussion. Participation and engagement of each student will be monitored during each class period.

\subsection{\textit{Lab journal}}
Every Friday after 4 pm, your electronic lab journal will be checked to ensure that it is up-to-date. The content of the lab journal will vary by week. Each week, a brief oral report of work done is required at the start of every class. Methods, data, and metadata should be uploaded to the lab journal during weeks when data are collected. Enough detail should be included in each laboratory journal entry to ensure independent reproduction and use of the data and/or task completed.

\subsection{\textit{Final report}}
The final report will consist of a project proposal for a project to be carried out in a subsequent semester. An outline of the report will be due on Friday, April 11, 2025 at 11:59pm via e-mail to Dr. Perkowski (see e-mail listed above). Students will also give brief 10-min presentations that summarize their project proposal during the final two meetings of the semester. The final report will be due on Friday, May 2, 2025 at 11:59pm via e-mail to Dr. Perkowski.\par

\section{Grading}
Participation and Engagement: 50\% \par
Lab journal: 25\% \par
Final report: 20\% \par
Final report presentation: 5\%\par

\section{Grading Scale}
A: $\geq$ 90\% \par
B: 80 – 90\% \par
C: 70 – 80\% \par
D: 60 – 70\% \par
F: $\leq$ 59.9\% \par

\section{Missing In-class Activities}
Students will be required to be in class to receive in-class activity and participation points. Please contact Dr. Perkowski if you plan to miss class for a university function \textit{prior to class}. If class is missed due to an illness, please let Dr. Perkowski know as soon as possible.

\section{Academic Integrity Statement}
Academic integrity is taking responsibility for one’s own class and/or course work, being individually accountable, and demonstrating intellectual honesty and ethical behavior. Academic integrity is a personal choice to abide by the standards of intellectual honesty and responsibility. Because education is a shared effort to achieve learning through the exchange of ideas, students, faculty, and staff have the collective responsibility to build mutual trust and respect. Ethical behavior and independent thought are essential for the highest level of academic achievement, which then must be measured. Academic achievement includes scholarship, teaching, and learning, all of which are shared endeavors. Grades are a device used to quantify the successful accumulation of knowledge through learning. Adhering to the standards of academic integrity ensures grades are earned honestly. Academic integrity is the foundation upon which students, faculty, and staff build their educational and professional careers [Texas Tech University Quality Enhancement Plan, Academic Integrity Task Force, 2010].

\section{Plagiarism Statement}
Texas Tech University expects students to “understand the principles of academic integrity and abide by them in all class and/or course work at the University” (OP 34.12.5). Plagiarism is a form of academic misconduct that involves (1) the representation of words,  ideas, illustrations, structure, computer code, other expression, or media of another as one's own and/or failing to properly cite direct, paraphrased, or summarized materials; or (2) self-plagiarism, which involves the submission of the same academic work more than once without the prior permission of the instructor and/or failure to correctly cite previous work written by the same student. Please review Section B of the TTU Student Handbook for more information related to other forms of academic misconduct, and contact your instructor if you have questions about plagiarism or other academic concerns in your courses. To learn more about the importance of academic integrity and practical tips for avoiding plagiarism, explore the resources provided by the TTU Library and the School of Law.

\section{AI Use}
The use of generative AI tools (such as ChatGPT) is strictly prohibited in this course for any purpose. Information gathered from AI cannot be used even with appropriate citation. Submission of AI-generated content (i.e., information, text, or images) as your own work is a violation of academic integrity and may result in referral to the Office of Student Conduct. Please contact Dr. Perkowski if you have questions regarding this course policy.

\section{ADA Statement}
Any student who, because of a disability, may require special arrangements in order to meet the course requirements should contact the instructor as soon as possible to make any necessary arrangements. Students should present appropriate verification from Student Disability Services during the instructor’s office hours. Please note: instructors are not allowed to provide classroom accommodations to a student until appropriate verification from Student Disability Services has been provided. For additional information, please contact Student Disability Services in Weeks Hall or call 806-742-2405.

\section{Religious Holy Day Statement}
"Religious holy day" means a holy day observed by a religion whose places of worship are exempt from property taxation under Texas Tax Code §11.20. A student who intends to observe a religious holy day should make that intention known in writing to the instructor prior to the absence. A student who is absent from classes for the observance of a religious holy day shall be allowed to take an examination or complete an assignment scheduled for that day within a reasonable time after the absence. A student who is excused under section 2 may not be penalized for the absence; however, the instructor may respond appropriately if the student fails to complete the assignment satisfactorily.

\section{Discrimination, Harassment, and Sexual Violence Statement}
Texas Tech University is committed to providing and strengthening an educational, working, and living environment where students, faculty, staff, and visitors are free from gender and/or sex discrimination of any kind. Sexual assault, discrimination, harassment, and other Title IX violations are not tolerated by the University. Report any incidents to the Office for Student Rights \& Resolution, (806)-742-SAFE (7233) or file a report online through the Title IX office. Faculty and staff members at TTU are committed to connecting you to resources on campus. Some of these available resources are:
\begin{itemize}
    \item \textbf{TTU Student Counseling Center}, 806- 742-3674: Provides confidential support on campus
    \item \textbf{TTU 24-hour Crisis Helpline}, 806-742-5555: Assists students who are experiencing a mental health or interpersonal violence crisis. If you call the helpline, you will speak with a mental health counselor.
    \item \textbf{Voice of Hope}, 806-763-7273: 24-hour hotline that provides support for survivors of sexual violence.
    \item \textbf{Risk, Intervention, Safety and Education (RISE) Office}, 806-742-2110: Provides a range of resources and support options focused on prevention education and student wellness.
    \item \textbf{Texas Tech Police Department}, 806-742- 3931: To report criminal activity that occurs on or near Texas Tech campus.
\end{itemize}

\section{Classroom Civility}
Texas Tech University is a community of faculty, students, and staff that enjoys an expectation of cooperation, professionalism, and civility during the conduct of all forms of university business, including the conduct of student–student and student–faculty interactions in and out of the classroom. Further, the classroom is a setting in which an exchange of ideas and creative thinking should be encouraged and where intellectual growth and development are fostered. Students who disrupt this classroom mission by rude, sarcastic, threatening, abusive or obscene language and/or behavior will be subject to appropriate sanctions according to university policy. Likewise, faculty members are expected to maintain the highest standards of professionalism in all interactions with all constituents of the university \newline (\href{www.depts.ttu.edu/ethics/matadorchallenge/ethicalprinciples.php}{www.depts.ttu.edu/ethics/matadorchallenge/ethicalprinciples.php}).

\section{Student Support}
The Office of Campus Access and Engagement works across Texas Tech University to foster, affirm, celebrate, engage, and strengthen all student communities. For more information about services, opportunities for participation, and ways in which Texas Tech can support your success in college, please contact (806) 742-7025.

\section{Food Insecurity}
If you are a student experiencing food or housing insecurity, and you believe this may affect your performance in the course, we strongly encourage you to reach out to the Raider Relief Advocacy and Resource Center for support. The Raider Relief Advocacy and Resource Center (RR-ARC) offers a central hub for students facing challenges related to basic needs. Through a broad network of campus and community resources, we aim to reduce the impact of financial, physical, and emotional difficulties, helping students thrive academically and personally. Units within the RR-ARC include Raider Relief Fund, Raider Red's Food Pantry (Doak Hall 117), and Fostering the Future. If you need assistance, please reach out to get support.

\section{Statement of Accommodation for Pregnant Students}
To support the academic success of pregnant and parenting students and students with pregnancy related conditions, the University offers reasonable modifications based on the student’s particular needs. Any student who is pregnant or parenting a child up to age 18 or has conditions related to pregnancy may contact Alex Faris, the Texas Tech University designated Pregnancy and Parenting Liaison, to discuss support available through the University. The Liaison can be reached by emailing \href{mailto:alfaris@ttu.edu}{alfaris@ttu.edu}. Should a student communicate with the instructor that they are pregnant or have a pregnancy related condition or may need additional resources related to pregnancy or parenting, the instructor will communicate that student’s information to the Title IX Coordinator, who will work with the student and others, as needed, to ensure equal access to the University’s education program or activity. 

For more information regarding supportive measures, please contact pregnancy \& parenting liaison Alex Faris (\href{mailto:alfaris@ttu.edu}{alfaris@ttu.edu} | 806.834.3420) or visit \newline \href{https://www.depts.ttu.edu/titleix/PregnancyandParenting/}{https://www.depts.ttu.edu/titleix/PregnancyandParenting/}. You can also visit \newline \href{https://www.depts.ttu.edu/titleix/PregnancyandParenting/}{https://www.depts.ttu.edu/titleix/PregnancyandParenting/} to submit a request to Alex Faris for assistance.


\newpage
\section*{Schedule of Topics and Deadlines by Week}
*subject to change based on student interests* \par

\subsection{Schedule of Topics}
01/22/2025 - Introductions, semester planning, and goal setting \par
01/29/2025 - Lab tour and discussion on plants and ecosystem services\par
02/05/2025 - A primer on plant ecophysiology and common measurements \par
02/12/2025 - Leaf and whole-plant responses to elevated CO$_2$ \par
02/19/2025 - Leaf and whole-plant responses to increased temperatures \par
02/26/2025 - Leaf and whole-plant responses to eutrophication \par
03/05/2025 - Plant responses to multiple global change factors \par
03/12/2025 - Plant-fungal mutualisms \par
03/19/2025 - NO CLASS - SPRING BREAK \par
03/26/2025 - Plant-fungal mutualisms and their role in shaping plant responses to global change \par
04/02/2025 - Discussion about research proposal, brainstorm session \par
04/09/2025 - Work session for proposal \par
04/16/2025 - Work session for proposal \par
04/23/2025 - Proposal presentations \par
04/30/2025 - Proposal presentations \par

\subsection{Important Due Dates}
04/09/2025 - Proposal Outline is due at 11:59pm \par
04/18/2025 - First draft of Proposal is due at 11:59pm \par
05/02/2025 - Final Proposal is due at 11:59pm \par
} %end font selection

\end{document} 
